\documentclass[a4paper,12pt,titlepage]{article}
\usepackage{amsmath}
\begin{document}

\title{Modelling/Geometric Transformations}

\author{Rajat Khanduja \\
	09010137 \\
	\and
	Bharat Khatri\\
	09010164}
\date{} %For empty date !

\maketitle

\tableofcontents
\pagebreak

\section{Modelling Transformations}
Changes in orientation, size and shape are accomplished with Geometric transformations that alter the co-ordinate description of objects. The basic geometric transformations are :- \\
\begin{itemize}
\item Rotation
\item Scaling
\item Translation
\item Reflection
\item Shear
\end{itemize}

\pagebreak
\section{Translation}
A \emph{translation} is applied to an object by repositioning it along a straight line path from one coordinate location to another.

\pagebreak
\section{Scaling}
\emph{Scaling} a coordinate means multiplying each of its components by a scalar. 


\begin {equation*}
	\begin{bmatrix}
		c & 0 & 0 & 0 \\
		0 & c & 0 & 0 \\
		0 & 0 & c & 0 \\
		0 & 0 & 0 & 1
	\end{bmatrix}
\end{equation*}

\pagebreak
\section{2D Rotation}
A \emph{2D Rotation} is applied to an object by repositioning it along a circular path in any of the three x-y, y-z or z-x planes.

\pagebreak
\section{Reflection}
A \emph{reflection} is a transformation that produces the mirror image of an object relative to an axis of reflection by rotating the object $180^\circ$ about the reflection axis. 
\pagebreak
\section{Shear}
A \emph{transformation} that distorts the shape of an object such that the transformed shape appears as if the object were composed of internal layers that had been caused to slide over each other.

\pagebreak
\end{document}
